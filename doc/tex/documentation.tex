\documentclass[listof=totoc]{article}
\usepackage[numbers]{natbib}					%use numbers for litrature references
\usepackage[footnote]{acronym}					%footnotes for acronyms
\usepackage[utf8]{inputenc}						%utf8 support (ö, ü, ä, ...)
\usepackage[ngerman]{babel}						%german names for table of contens and litratur
\usepackage[nottoc,notlot,notlof]{tocbibind}	%adds bibliographie to the table of contens
\usepackage{a4wide}								%use more space of the page
\usepackage{hyperref}							%clickable links
\hypersetup{
	colorlinks,
	citecolor=black,
	filecolor=black,
	linkcolor=black,
	urlcolor=black
}

\usepackage{listings}
\usepackage{xcolor}

\definecolor{gray}{rgb}{0.4,0.4,0.4}
\definecolor{background}{HTML}{EEEEEE}
\definecolor{darkblue}{rgb}{0.0,0.0,0.6}
\definecolor{cyan}{rgb}{0.0,0.6,0.6}
\definecolor{lightgray}{rgb}{.9,.9,.9}

\lstset{
  basicstyle=\ttfamily,
  columns=fullflexible,
  showstringspaces=false,
  commentstyle=\color{gray}\upshape
}

\lstdefinelanguage{XML}
{
  backgroundcolor=\color{background},
  morestring=[b]",
  morestring=[s]{>}{<},
  morecomment=[s]{<?}{?>},
  stringstyle=\color{black},
  identifierstyle=\color{darkblue},
  keywordstyle=\color{cyan},
  morekeywords={xmlns,version,type}% list your attributes here
}

\usepackage{listings}
\usepackage{xcolor}

\colorlet{punct}{red!60!black}
\definecolor{background}{HTML}{EEEEEE}
\definecolor{delim}{RGB}{20,105,176}
\colorlet{numb}{magenta!60!black}


\lstdefinelanguage{json}{
	basicstyle=\normalfont\ttfamily,
%	numbers=left,
%	numberstyle=\scriptsize,
%	stepnumber=1,
%	numbersep=8pt,
	showstringspaces=false,
	breaklines=true,
	backgroundcolor=\color{background},
	literate=
	*{0}{{{\color{numb}0}}}{1}
	{1}{{{\color{numb}1}}}{1}
	{2}{{{\color{numb}2}}}{1}
	{3}{{{\color{numb}3}}}{1}
	{4}{{{\color{numb}4}}}{1}
	{5}{{{\color{numb}5}}}{1}
	{6}{{{\color{numb}6}}}{1}
	{7}{{{\color{numb}7}}}{1}
	{8}{{{\color{numb}8}}}{1}
	{9}{{{\color{numb}9}}}{1}
	{:}{{{\color{punct}{:}}}}{1}
	{,}{{{\color{punct}{,}}}}{1}
	{\{}{{{\color{delim}{\{}}}}{1}
	{\}}{{{\color{delim}{\}}}}}{1}
	{[}{{{\color{delim}{[}}}}{1}
	{]}{{{\color{delim}{]}}}}{1},
}
\usepackage{listings}
\usepackage{xcolor}

\definecolor{background}{HTML}{EEEEEE}
\definecolor{lightgray}{rgb}{.9,.9,.9}
\definecolor{darkgray}{rgb}{.4,.4,.4}
\definecolor{purple}{rgb}{0.65, 0.12, 0.82}

\lstdefinelanguage{JavaScript}{
	keywords={typeof, new, true, false, catch, function, return, null, catch, switch, var, if, in, while, do, else, case, break},
	keywordstyle=\color{blue}\bfseries,
	ndkeywords={class, export, boolean, throw, implements, import, this},
	ndkeywordstyle=\color{darkgray}\bfseries,
	identifierstyle=\color{black},
	sensitive=false,
	comment=[l]{//},
	morecomment=[s]{/*}{*/},
	commentstyle=\color{purple}\ttfamily,
	stringstyle=\color{red}\ttfamily,
	morestring=[b]',
	morestring=[b]"
}

\lstset{
	language=JavaScript,
	backgroundcolor=\color{background},
	extendedchars=true,
	basicstyle=\footnotesize\ttfamily,
	showstringspaces=false,
	showspaces=false,
	numbers=left,
	numberstyle=\footnotesize,
	numbersep=9pt,
	tabsize=2,
	breaklines=true,
	showtabs=false,
	captionpos=b
}



%opening
\title{RESTful API mit node.js}
\author{AUTHOR}

\begin{document}
	\maketitle
	\begin{abstract}
		\begin{center}
			Dieses Dokument beschriebt die Theorie hinter \ac{REST} und \\
			die Implementierung einer RESTful \ac{API} mithilfe von node.js
		\end{center}
	\end{abstract}
	\newpage
	
	\tableofcontents 
	\lstlistoflistings
	
	\newpage
	
	\section{Allgemeines über \ac{REST}}
	\ac{REST} ist ein Programmierparadigma und wurde von Roy Fielding in seiner Dissertation\cite{REST_THESIS} spezifiziert.
	\newline \newline
	Die wichtigsten Eigenschaften\cite[p.~2]{RESTBOOK} sind:
	\begin{itemize}  
		\item Alles ist eine Ressource und jede ist mit einer eindeutigen Adresse identifizierbar (\ac{URI})
		\item Verwendung von Standard-\ac{HTTP} Methoden\footnote{Keine strenge Voraussetzung für \ac{REST}, in der Praxis wird aber hauptsächlich \ac{HTTP} verwendet}
		\item Ressourcen können mehrerer Repräsentationen besitzen
		\item Zustandslose Kommunikation
		\item  \ac{HATEOAS}
	\end{itemize}
	\subsection{Ressourcen}
	Die Ressourcen sind unter einer eindeutigen Adresse (\ac{URI}) erreichbar und können unter verschiedenen Repräsentationen dargestellt werden. Siehe Kapitel \ref{subsec:Repraesentation}.
	\newline \newline
	Zum Beispiel kann ein Ressource folgende \ac{URI} haben "http://ipaddress:port/path/Ressource"
	%Die Repräsentation kann jedoch in verschiedenen Formaten erfolgen.
	\subsection{\ac{HTTP}-Methoden}
	Die Verwendung von \ac{HTTP}-Methoden ist keine strickte Voraussetzung für REST.
	In der Praxis wird jedoch hauptsächlich \ac{HTTP} verwendet für REST. Deshalb wird hier auch nur auf die Implementierung mit \ac{HTTP}-Methoden eingegangen.
	\newline \newline
	Es sind folgende Operationen laut RFC2616\cite{RFC2616} im \ac{HTTP}-Protokoll möglich:
	\begin{itemize} 
		\setlength{\itemsep}{0pt} 
		\item GET
		\item POST
		\item PUT
		\item DELETE
		\item HEAD
		\item OPTIONS
		\item TRACE
		\item CONNECT
	\end{itemize}
	\begin{center}
	Für \ac{REST} werden die folgenden 4 \ac{CRUD}-Operationen verwendet. 
	\newline \newline
	\begin{tabular}{|c|c|}
		\hline 
		Create & POST \\ 
		\hline 
		Read & GET \\ 
		\hline 
		Update & PUT \\ 
		\hline 
		Delete & DELETE \\ 
		\hline 
	\end{tabular} 
	\end{center}
	\newpage
	\subsubsection{POST - Create}
	Erzeugt eine neue Ressource.
	\newline \newline
	Beispiel:
	\begin{tabular}{|c|c|c|c|}
		\hline 
		Method & URI & Sent Data & Response \\ 
		\hline 
		POST & /Persons & Ressource Data & /Persons/123 \\ 
		\hline 
	\end{tabular} 
	\newline
	\subsubsection{GET - Read}
	Um eine Ressource anzuzeigen wird die GET-Operation verwendet. 
	\newline \newline
	Beispiel:
	\begin{tabular}{|c|c|c|c|}
		\hline 
		Method & URI & Sent Data & Response \\ 
		\hline 
		GET & /Persons/123 & none & Resource Data \\ 
		\hline 
	\end{tabular} 
	\subsubsection{PUT - Update}
	Verändert eine bestehende Ressource.
	\newline \newline
	Beispiel:
	\begin{tabular}{|c|c|c|c|}
		\hline 
		Method & URI & Sent Data & Response \\ 
		\hline 
		PUT & /Persons/123 & Ressource Data & none \\ 
		\hline 
	\end{tabular} 
	\subsubsection{DELETE - Delete}
	Löscht eine Ressource.
	\newline \newline
	Beispiel:
	\begin{tabular}{|c|c|c|c|}
		\hline 
		Method & URI & Sent Data & Response \\ 
		\hline 
		DELETE & /Persons/123 & none & none \\ 
		\hline 
	\end{tabular} 
	\subsection{Repräsentation}\label{subsec:Repraesentation}
	Eine Ressource kann unterschiedliche Repräsentationen besitzen.
	Als Beispiel wird als eine Ressource eine Person betrachtet. Diese enthält folgende Daten:
	\begin{itemize} 
		\setlength{\itemsep}{0pt} 
		\item Vorname
		\item Nachname
		\item Telefonnummer
	\end{itemize}
	Die Person kann in verschiedenen Formaten dargestellt werden. 
	In \ac{JSON} würde die Repräsentation so aussehen:
	
	
	\begin{lstlisting}[language=json,caption=\ac{JSON}-Repräsentation (application/json)]
		{"vorname":"Hans","nachname":"Mueller","telefonnummer":"0680123456789"}
	\end{lstlisting}
	
	Die selbe Person kann allerdings auch in \ac{XML} dargestellt werden:
	\begin{lstlisting}[language=XML,caption=\ac{XML}-Repräsentation (application/xml)]
	<?xml version="1.0" encoding="UTF-8"?>
	<root>
		<nachname>Mueller</nachname>
		<telefonnummer>0680123456789</telefonnummer>
		<vorname>Hans</vorname>
	</root>
	\end{lstlisting}
	\newpage
	
	
	\subsection{Zustandslose Kommunikation}
	Eine Vorraussetzung für eine \ac{REST}ful \ac{API} ist die Zustandslosigkeit. Weder Server noch Client sollten Zustandsinformationen speicher. Das heißt Anfragen an den Server sollten in sich geschlossen sein. Dies hat den Vorteil der einfachen Skalierbarkeit, das heißt Anfragen können an unterschiedliche Server verteilt werden. \cite[p.~6]{RESTBOOK}
	
	\subsection{Hypermedia as the Engine of Application State}
	\ac{HATEOAS} ist laut Roy Fielding eine Voraussetzung für eine \ac{REST}ful \ac{API}.\cite{HATEOAS_ROY}
	Dazu müssen Ressourcen Links (Hypermedia) zu anderen Ressourcen enthalten können. Es sollte somit möglich sein nur durch Kenntnis der Root-\ac{URI} zu allen möglichen \ac{URI}s weiter zu navigieren. 
	Somit kann man von einer Ressource zur Nächsten navigieren. 
	\newline
	Oft werden jedoch auch fälschlicherweise \ac{API}s als \ac{REST}ful bezeichnet die nicht \ac{HATEOAS} verwenden.
	\newpage
	
	\section{Beispielprogramm}
	Da \ac{REST} nur ein Programmierparadigma ist, hat man sehr viel Möglichkeiten bei der Implementierung. Programmiersprache, Repräsentation der Daten sowie Protokoll für die Datenübertragung sind nicht vorgegeben durch \ac{REST}. Deshalb ist der erste Schritt für die Implementierung notwendig zu entscheiden welche Tools verwendet werden sollen.
	\subsection{Programmiersprache}
	Als Programmiersprache wird in diesem Beispiel Node.js in der Version v6.11.4 verwendet.
	\begin{quote}
		"Node.js$^{\tiny{\textregistered}}$ ist eine JavaScript-Laufzeitumgebung, die auf Chromes V8 JavaScript-Engine basiert. Durch ein Event-basiertes, blockierungsfreies I/O-Modell ist Node.js schlank und effizient."
		\footnote{https://nodejs.org/de/}
	\end{quote}
	\subsection{Daten Repräsentation}
	Da Ressourcen durch unterschiedliche Repräsentationen dargestellt werden können, könnten prinzipiell auch mehrere Repräsentationen vom Beispielprogramm unterstützt werden. Der Einfachheit halber wird allerdings nur ein Format verwendet.
	\ac{JSON} bietet sich aufgrund seiner weiten Verbreitung und Kompatibilität mit JavaScript an. Doch \ac{JSON} alleine bietet keine Verlinkungen zwischen Datensätzen an und erfüllt deshalb nicht \ac{HATEOAS}. Deshalb gibt es zahlreiche Formate die auf \ac{JSON} basieren aber zusätzlich auch \ac{HATEOAS} unterstützen. \cite{HATEOAS_FORMATS}
	\begin{itemize} 
		\setlength{\itemsep}{0pt} 
		\item \ac{JSON-LD}
		\item \ac{HAL}
		\item Collection+\ac{JSON}
		\item \ac{SIREN}
	\end{itemize}
	Für dieses Beispiel wurde Collection+\ac{JSON}\cite{CJ} ausgewählt.
	\subsection{Datenbank}
	Im Normalfall werden die Ressourcen in einer Datenbank abgespeichert. Da es in diesem Beispielprogramm aber hauptsächlich um die Implementierung einer \ac{REST}ful-\ac{API} geht,
	wird auf eine Datenbank verichtet und es werden die Ressourcen nur als Variablen im Programm gespeichert. Das heißt, falls das Programm beendet wird sind alle Daten verloren.
	Dieses Beispielprogramm kann somit ohne Anpassungen nicht in einer produktiven System eingesetzt werden! Oft werden in Produktivsystemen \ac{noSQL} Datenbanken verwendet.
	
	\newpage
	\subsubsection{Collection+\ac{JSON}}
	Wenn beispielsweise die Root-\ac{URI} aufgerufen wird, liefert der Server folgende Antwort. 
	\begin{lstlisting}[language=json,caption=Collection+\ac{JSON}-Repräsentation (application/vnd.collection+json)]
	{
		"collection": {
			"version": "1.0",
			"href": "http://localhost:1337",
			"items": [],
			"links": [
				{
					"rel": "home",
					"href": "http://localhost:1337",
					"promt": ""
				},
				{
					"rel": "persons",
					"href": "http://localhost:1337/persons",
					"promt": ""
				},
				{
					"rel": "adresses",
					"href": "http://localhost:1337/adresses",
					"promt": ""
				}
			],
			"queries": [],
			"templates": []
		}
	}
	\end{lstlisting}
	Man sieht, dass man nun 2 Möglichkeiten hat weiter zu navigieren.
	\newline
	"'http://localhost:1337/persons"' oder "'http://localhost:1337/adresses"'
	
	\subsection{Protokoll}
	Als Protokoll wird \ac{HTTP} verwendet, weil es das am meisten verwendete für \ac{REST} ist.
	\newpage
	
	\lstset{numbers=none}	%disable numbering for listings
	\subsection{\ac{API} Beschreibung}
	\subsubsection{POST /persons}
	Um eine neue Person zu Erstellen wird die \ac{URI} "'http://localhost:1337/persons"' mit der POST-Methode aufgerufen.
	\newline
	
	\begin{tabular}{|c|p{10cm}|}
		\hline 
		Methode & POST \\ 
		\hline 
		\ac{URI} & "'http://localhost:1337/persons"' \\ 
		\hline 
		Sent-Body &  
		\begin{lstlisting}[language=json]
		{
			"template": {
				"data": [
				{
					"name": "givenname",
					"value": "Allen",
					"promt": "givenname"
				},
				{
					"name": "familyname",
					"value": "David",
					"promt": "familyname"
				},
				{
					"name": "phonenumber",
					"value": "0699 615 55 82",
					"promt": "phonenumber"
				},
				{
					"name": "address",
					"value": "http://localhost:1337/address/0",
					"promt": "Link to adress"
				}
				]
			}
		}
		\end{lstlisting} \\ 
		\hline 
		Response-Header & Location: http://localhost:1337/persons/4 \\ 
		\hline 
		Response-Body & - \\ 
		\hline 
		Response-Code & 201 Created \\ 
		\hline 
		cURL-Example & 	
		\begin{verbatim}
			curl --request POST \
			--url http://localhost:1337/persons/ \
			--header 'Content-Type: application/json' \
			--data '{...}'
		\end{verbatim}
		\\ 
		\hline 
	\end{tabular} 
	\newpage


	\subsubsection{POST /addresses}
	Um eine neue Adresse zu Erstellen wird die \ac{URI} "'http://localhost:1337/addresses"' mit der POST-Methode aufgerufen.
	\newline
	\begin{tabular}{|c|p{10cm}|}
		\hline 
		Methode & POST \\ 
		\hline 
		\ac{URI} & "'http://localhost:1337/adresses"' \\ 
		\hline 
		Sent-Body &  
		\begin{lstlisting}[language=json]
		{
			"template": {
				"data": [
				{
					"name": "country",
					"value": "Austria",
					"promt": "country"
				},
				{
					"name": "state",
					"value": "Carinthia",
					"promt": "state"
				},
				{
					"name": "zipCode",
					"value": "9560",
					"promt": "zipCode"
				},
				{
					"name": "city",
					"value": "Powirtschach",
					"promt": "city"
				},
				{
					"name": "streetAddress",
					"value": "Auenweg 71",
					"promt": "streetAddress"
				}
				]
			}
		}
		\end{lstlisting} \\ 
		\hline 
		Response-Header & Location: http://localhost:1337/addresses/4 \\ 
		\hline 
		Response-Body & - \\ 
		\hline 
		Response-Code & 201 Created \\ 
		\hline 
		cURL-Example &  
		\begin{verbatim}
			curl --request POST \
			--url http://localhost:1337/addresses/ \
			--header 'Content-Type: application/json' \
			--data '{...}'
		\end{verbatim}
		\\ 
		\hline 
	\end{tabular} 
	\newpage
	
	\subsubsection{GET /persons}
		Um eine neue Adresse zu Erstellen wird die \ac{URI} "'http://localhost:1337/addresses"' mit der POST-Methode aufgerufen.
	\newline
	\begin{tabular}{|c|p{10cm}|}
		\hline 
		Methode & POST \\ 
		\hline 
		\ac{URI} & "'http://localhost:1337/adresses"' \\ 
		\hline 
		Sent-Body & - \\ 
		\hline 
		Response-Header & - \\ 
		\hline 
		Response-Body & - \\ 
		\hline 
		Response-Code & 200 OK \\ 
		\hline 
		cURL-Example &  
		\begin{verbatim}
		curl --request POST \
		--url http://localhost:1337/addresses/ \
		--header 'Content-Type: application/json' \
		--data '{...}'
		\end{verbatim}
		\\ 
		\hline 
	\end{tabular} 
	\newpage
	\subsubsection{GET /addresses}
	\subsubsection{PUT /persons}
	\subsubsection{PUT /addresses}
	\subsubsection{DELETE /persons}
	\subsubsection{DELETE /addresses}
	\subsection{Source Code}
	\newpage
	
	\setcounter{secnumdepth}{0}	%disable numbering of sections
	\section{Abkürzungen}
	\begin{acronym}[HATEOAS]
		\acro{REST}{Representational State Transfer}
		\acro{API}{Application Programming Interface}
		\acro{URI}{Unique Adress Identifier}
		\acro{HTTP}{Hypertext Transfer Protocol}
		\acro{HATEOAS}{Hypermedia as the Engine of Application State}
		\acro{CRUD}{Create Read Update Delete}
		\acro{JSON}{JavaScript Object Notation}
		\acro{XML}{Extensible Markup Language}
		\acro{JSON-LD}{\ac{JSON} for Linked Documents}
		\acro{HAL}{Hypertext Application Language}
		\acro{SIREN}{Structured Interface for Representing Entities}
		\acro{SQL}{Structured Query Language}
		\acro{noSQL}{Not only \ac{SQL}}
	\end{acronym}
	\newpage

	

	\bibliographystyle{plainnat}
	\bibliography{literatur}
\end{document}
