\documentclass[]{article}
\usepackage[numbers]{natbib}					%use numbers for litrature references
\usepackage[footnote]{acronym}					%footnotes for acronyms
\usepackage[utf8]{inputenc}						%utf8 support (ö, ü, ä, ...)
\usepackage[ngerman]{babel}						%german names for table of contens and litratur
\usepackage[nottoc,notlot,notlof]{tocbibind}	%adds bibliographie to the table of contens
\usepackage{a4wide}


%opening
\title{RESTful API mit node.js}
\author{AUTHOR}

\begin{document}
	\maketitle
	\begin{abstract}
		\begin{center}
			Dieses Dokument beschriebt die Theorie hinter \ac{REST} und \\
			die Implementierung einer RESTful \ac{API} mithilfe von node.js
		\end{center}
	\end{abstract}
	\newpage
	
	\tableofcontents 
	\newpage
	
	\section{Allgemeines über \ac{REST}}
	\ac{REST} ist ein Programmierparadigma und wurde von Roy Fielding in seiner Dissertation\cite{REST_THESIS} spezifiziert.
	\newline \newline
	Die wichtigsten Eigenschaften\cite[p.~2]{RESTBOOK} sind:
	\begin{itemize}  
		\item Alles ist eine Ressource und jede ist mit einer eindeutigen Adresse identifizierbar (\ac{URI})
		\item Verwendung von Standard-\ac{HTTP} Methoden\footnote{Keine strenge Voraussetzung für \ac{REST}, in der Praxis wird aber hauptsächlich \ac{HTTP} verwendet}
		\item Ressourcen können mehrerer Repräsentationen besitzen
		\item Zustandslose Kommunikation
		\item  \ac{HATEOAS}
	\end{itemize}
	\subsection{Ressourcen}
	Die Ressourcen sind unter einer eindeutigen Adresse (\ac{URI}) erreichbar und können unter verschiedenen Repräsentationen dargestellt werden:
	\newline \newline
	Zum Beispiel kann ein Ressource folgende \ac{URI} haben "http://ipadress:port/path/Ressource
	%Die Repräsentation kann jedoch in verschiedenen Formaten erfolgen.
	\subsection{\ac{HTTP}-Methoden}
	Die Verwendung von \ac{HTTP}-Methoden ist keine strickte Voraussetzung für REST.
	In der Praxis wird jedoch hauptsächlich \ac{HTTP} verwendet für REST. Deshalb wird hier auch nur auf die Implementierung mit \ac{HTTP}-Methoden eingegangen.
	\newline \newline
	Es sind folgende Operationen laut RFC2616\cite{RFC2616} im \ac{HTTP}-Protokoll möglich:
	\begin{itemize} 
		\setlength{\itemsep}{0pt} 
		\item GET
		\item POST
		\item PUT
		\item DELETE
		\item HEAD
		\item OPTIONS
		\item TRACE
		\item CONNECT
	\end{itemize}
	\begin{center}
	Für \ac{REST} werden die folgenden 4 \ac{CRUD}-Operationen verwendet. 
	\newline \newline
	\begin{tabular}{|c|c|}
		\hline 
		Create & POST \\ 
		\hline 
		Read & GET \\ 
		\hline 
		Update & PUT \\ 
		\hline 
		Delete & DELETE \\ 
		\hline 
	\end{tabular} 
	\end{center}
	\newpage
	asdasd
	
	\newpage
	
	\setcounter{secnumdepth}{0}	%disable numbering of sections
	\section{Abkürzungen}
	\begin{acronym}[HTTP]
		\acro{REST}{Representational State Transfer}
		\acro{API}{Application Programming Interface}
		\acro{URI}{Unique Adress Identifier}
		\acro{HTTP}{Hypertext Transfer Protocol}
		\acro{HATEOAS}{Hypermedia as the Engine of Application State}
		\acro{CRUD}{Create Read Update Delete}
	\end{acronym}
	\newpage
	

	\bibliographystyle{plainnat}
	\bibliography{literatur}
\end{document}
